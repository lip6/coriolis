\subsubsection{Name}
    Alias -- A net has an "alias name"

\subsubsection{Synopsys}

\verb-myNet.Alias ( net )-

\subsubsection{Description}

This method is applied to a net. This net has an "alias name".

\subsubsection{Parameters}

\begin{itemize}
    \item \verb-net- : a net which is going to be an alias for the net which this method is applied to
\end{itemize}

\subsubsection{Example}

\begin{verbatim}
class myripple ( Model ) :
    
  def Interface ( self ) :
    self.a    = LogicIn  (    "a", 4 )
    self.b    = LogicIn  (    "b", 4 )

    self.cin  = LogicIn  (  "cin", 1 )

    self.sout = LogicOut ( "sout", 4 )

    self.cout = LogicOut ( "cout", 1 )

    self.vdd  = VddIn ( "vdd" )
    self.vss  = VddIn ( "vss" )

  def Netlist ( self ) :
    c_temp = Signal ( "c_temp", 5 )
    
    self.cin.Alias  ( c_temp[0] )
    self.cout.Alias ( c_temp[4] )
          
    for i in range ( 4 ) :
      Inst ( "Fulladder"
           , map = { 'a'    : self.a[i]
                   , 'b'    : self.b[i]
                   , 'cin'  : c_temp[i]
                   , 'sout' : self.sout[i]
                   , 'cout' : c_temp[i+1]
                   , 'vdd'  : self.vdd
                   , 'vss'  : self.vss
                   }
           )
\end{verbatim}

\indent The net \verb-cin- has the alias \verb-c_temp[0]- and the net cout has the alias \verb-c_temp[4]-. Thanks to this method, all the instanciations can be done in one unique \verb-for- loop.
     
\subsubsection{See Also}

\hyperref[ref]{\emph{Introduction}}{}{Introduction}{secintroduction}
\hyperref[ref]{\emph{Nets}}{}{Nets}{secnet}
\hyperref[ref]{\emph{Extend}}{}{Extend}{secextend}
\hyperref[ref]{\emph{Cat}}{}{Cat}{seccat}
